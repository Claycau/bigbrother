\chapter{Improved Ensemble}
Discuss our improved ensemble method here

\subsection{Activity recog}
discuss activity recognition and its importance here

\subsection{HMM vs TSMG}
discuss why we are not using hidden markov models as stated in the proposal

\subsection{TSMG derivation}

A mixture of Gaussians is a strongly supported stochastic data clustering technique used in activity recognition.  Traditionally, a mixture of Gaussians is implemented for either a one dimensional time series (CITE PAPER TO SUPPORT THIS) or for data vectors with no time element.  Here we combine the two approaches, creating a mixture of Gaussians for multi-dimensional time series data.  While this approach has not yet been implemented, based on the success of mixture of Gaussians in other domains, we expect good results.

The goal of mixture of Gaussians is to find a set of models which will maximize the log likelihood of the parameters of some models to the dataset.  Given dataset $\{x^{(i)}\}$ we maximize
\begin{equation}
\ell(\theta) = \sum_{i = 1}^{\bf M}log\{p(x^{(i)}|\theta)\}
\end{equation}
\noindent 
where ${\bf M}$ is the total number of time series instances.

The expectation maximization (EM) algorithm is commonly used to maximize dataset likelihood.  To use this algorithm we need to define a set of variables
\begin{equation}
w_{k}^{(i)} = p(z = k|x^{(i)})
\end{equation}
\noindent
where ${\bf K}$ is the total number of Gaussians to train and $k$ is an index of ${\bf K}$.  

The general equation for the likelihood of the models is: 
\begin{equation}
\label{eq:em_likelihood}
\ell(\theta|x) = \sum_{i = 1}^{{\bf M}}\sum_{k = 1}^{{\bf K}}w_{k}^{(i)}\log \{ \frac{p(x^{(i)}|z=k)p(z = k)}{w_{k}^{(i)}} \}
\end{equation}

In the traditional mixture of Gaussians algorithm each model is ostensibly a Gaussian.  To make this algorithm work with multi-dimensional time series, we define the models instead by
\begin{equation}
\label{eq:model}
p(x^{(i)}|z = k) = \prod_{n = 1}^{{\bf N}}\mathcal{N}_{n}(x^{(i)})
\end{equation}
\noindent
where ${\bf N}$ is the length of each time series instance.  Thus our model for each time series is ${\bf N}$ independent multivariate Gaussians.

Combining equations~\ref{eq:em_likelihood} and~\ref{eq:model} gives the following log likelihood
\begin{equation}
\label{eq:em_combined}
\ell(\theta|x) = \sum_{i = 1}^{{\bf M}}\sum_{k = 1}^{{\bf K}}w_{k}^{(i)}\{ \log\frac{p(z = k)}{w_{k}} + \sum_{n = 1}^{{\bf N}} \log \mathcal{N}_{n}(x^{(i)})\}
\end{equation}

\textbf{E-Step}
The E-step hardly changes from the traditional EM mixture of Gaussians algorithm.  We simply need to calculate 
\begin{equation}
w^{(i)}_{k} = p(z = k|x^{(i)})
\end{equation}

\textbf{M-Step}
For the maximization step, it is assumed that we know the values of $w_{k}^{(i)}$.  Thus, we need to maximize equation~\ref{eq:em_combined} with respect to $\mu$,  $\Sigma$, and $\theta$.
The results of these maximizations are given below:
\begin{equation}
\theta_{k} = \frac{1}{{\bf M}}\sum_{i = 1}^{{\bf M}}w_{k}^{(i)}
\end{equation}
\begin{equation}
\mu_{k, n} = \frac{\sum_{i = 1}^{{\bf M}}w_{k}^{(i)}x^{(i)}_{n}}{\sum_{i = 1}^{{\bf M}}w_{k}^{(i)}}
\end{equation}
\begin{equation}
\Sigma_{k, n} = \frac{\sum_{i = 1}^{{\bf M}}w_{k}^{(i)}(x^{(i)} - \mu_{k, n})(x^{(i)} - \mu_{k, n})^{\mathrm{T}}}{\sum_{i = 1}^{{\bf M}}w_{k}^{(i)}}
\end{equation}

\subsection{Clustering activities}
TODO DISCUSS OUR CLUSTERING HERE

\subsection{FINDING POTENTIAL DATAPOINTS}
TODO DISCUSS OUR POTENTIAL DATAPOINTS HERE

\subsection{HMM Clusters}
DICUSS OUR HMM CLUSTERING METHOD AND REPRESENTATION HERE

\subsection{On the limitations of BCF}
DISCUSS OUR NEED FOR ANOTHER METHOD HERE

\subsection{Improved Ensemble Derivation}
Derive our improved ensemble method here.

\subsection{RESULTS ON Improved Ensemble Derivation}
Discuss and display the results here




